%-------------------------------------------------------------------------------
%	PACKAGES AND OTHER DOCUMENT CONFIGURATIONS
%-------------------------------------------------------------------------------
\documentclass[english,counters by chapter]{uniud}
\pdfoutput=1

\usepackage[T1]{fontenc}
\usepackage[utf8]{inputenc}

\usepackage{subfiles}

% section numbering
% depth 3: subsubsections but not subsubsubsections
\setcounter{secnumdepth}{3}

% table of contents (TOC)
% depth 2: sections and subsections
\setcounter{tocdepth}{2}

\hypersetup{
	% do not highlight hyperlinks such as citations
	hidelinks,
	% link on both page and title in TOC
	linktoc=all
}

\usepackage{lipsum,blindtext} % generates demo text

%-------------------------------------------------------------------------------
%	FOREIGN
%-------------------------------------------------------------------------------
\usepackage{babel}
\usepackage[all]{foreign}
\defasforeign[mutatismutandis]{mutatis mutandis}
\defasforeign[Mutatismutandis]{Mutatis mutandis}
\defasforeign[criteria]{criteria}
\defasforeign[criterium]{criterium}


%-------------------------------------------------------------------------------
%	BIBLATEX
%-------------------------------------------------------------------------------

\usepackage[
	backend=bibtex, %backend=biber
	% numeric citations with compression of consecutive numers
	% e.g. [1-4] instead of [1,2,3,4]
	style=numeric-comp,
	% use given name initials for authors
	giveninits=true,
	% sort references by author names, year and then title
	sorting=nyt,
	% limits the number of displayed authors in the bibliography (default is 3)
	maxbibnames=99,
	% do not print urls or doi and isbn identifiers
	doi=false, 
	isbn=false,
	url=false
	]{biblatex}
\addbibresource{dummy.bib} %biblio sources

%-------------------------------------------------------------------------------
%	TiKZ
%-------------------------------------------------------------------------------
\usepackage{tikz}

%-------------------------------------------------------------------------------
%	COMMANDS
%-------------------------------------------------------------------------------

%-------------------------------------------------------------------------------
%	TITLE PAGE
%-------------------------------------------------------------------------------
\uniudset{
	department={\uniudname{dima}}, %default
	course={\uniudname{dima/informatica}}, % default
	% mandatory
	degree=phd,
	candidate name={Grace ``Amazing'' Hopper},
	academic year={1933-34},
	title={New Types of Irreducibility Criteria},
	supervisor={{\O}ystein Ore},
	% optional
	candidate email={hopper.grace@spes.uniud.it},
	candidate address={
		Dipartimento di Matematica, Informatica e Fisica\\
		Università degli Studi di Udine\\
		Via delle Scienze, 206\\
		33100 Udine\\
		Italia
	},
	candidate use gender=feminine,
	%cosupervisor={CoSupervisor},
	dedication={Dedication},
	inside cover note={\today},
}

\hypersetup{
  pdftitle={\pgfkeysvalueof{/uniud/title}},
  pdfauthor={\pgfkeysvalueof{/uniud/candidate name}},
}

\begin{document}

%-------------------------------------------------------------------------------
%	FRONT MATTER
%-------------------------------------------------------------------------------

\frontmatter

\maketitle
 
% \MakeTextLowercase{\expandafter\expandafter\expandafter\uniudname{uniud}}
% 
% \MakeTextLowercase{\pgfkeysvalueof{/uniud/title}}

\chapter*{Acknowledgments}

\blindtext

\chapter*{Abstract}

\blindtext

\cleardoublepage

\tableofcontents*

%-------------------------------------------------------------------------------
%	MAIN MATTER
%-------------------------------------------------------------------------------

\mainmatter

\chapter{Chapter}
\blindmathtrue

\blindtext

\section{Section}

\blindtext

\subsection{Subsection}

\blindtext

\subsubsection{Subsubsection}

\blindtext

\paragraph{Paragraph}

\blindtext

\subparagraph{Subparagraph}

\blindtext

\section{Lists}


Itemize
\begin{itemize}
\item First item text
\item Second item text
\item Third item text
\end{itemize}
Enumerate
\begin{enumerate}
\item First item text
\item Second item text
\item Third item text
\end{enumerate}
Description
\begin{description}
\item[First] 
	Lorem ipsum dolor sit amet, consectetuer adipiscing elit. Etiam lobortis facilisis sem. Nullam nec mi et neque pharetra sollicitudin. Praesent imperdiet mi nec ante.
\item[Second] 
	 Donec ullamcorper, felis non sodales commodo, lectus velit ultrices augue, a dignissim nibh lectus placerat pede.
\item[Third] 
	Vivamus nunc nunc, molestie ut, ultricies vel, semper in, velit. Ut porttitor. Praesent in sapien.
\end{description}


\section{Math}

\begin{theorem}
	\label{thm:first}
	Theorem
\end{theorem}
\begin{proof}
	Proof
\end{proof}

\begin{corollary}
	\label{cor:first}
	Corollary
\end{corollary}

\begin{proposition}
	\label{prop:first}
	Proposition
\end{proposition}

\begin{lemma}
	\label{lem:first}
	Lemma
\end{lemma}

\begin{definition}
	\label{def:first}
	Definition
\end{definition}

\begin{remark}
	\label{rem:first}
	\blindtext[1]
\end{remark}

\begin{figure}[t]
	\begin{center}
	\begin{tikzpicture}
		\draw (0,0) circle (4cm);
	\end{tikzpicture}
	\end{center}
	\caption{This figure has a very long caption\dots let's see if it comes out nice\dots}
	\label{fig:first}
\end{figure}

\section{Cross referencing and citations}

Here are some references and citations: \Cref{def:first,fig:first}  \cite{article,book,booklet,conference,inbook,incollection,manual,mastersthesis,misc,phdthesis,proceedings,techreport,unpublished}.

\subfile{subfile-example}

%-------------------------------------------------------------------------------
%	BACK MATTER
%-------------------------------------------------------------------------------

%\backmatter

\printbibliography

\appendix

\chapter{Appendix}
\lipsum[1-7]

\end{document}