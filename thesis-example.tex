%-------------------------------------------------------------------------------
%	PACKAGES AND OTHER DOCUMENT CONFIGURATIONS
%-------------------------------------------------------------------------------
\documentclass[italian,english]{uniud}
% The class supports only italian and english and follows the conventions of 
% the package 'babel'. Both italian and english can be loaded in order to 
% support bilingual documents. In this case, the leanguage loaded last is the 
% main one i.e. the one used for localising the document strings (e.g. theorem 
% or definition). The main language can be changed locally (e.g. to provide 
% translated abstracts or acknowledgments) via babel's 'otherlanguage' 
% environment. 
% Paper is set to A4.
% Reference font size is 11pt.

% font encoding
\usepackage[T1]{fontenc}
\usepackage[utf8]{inputenc}

% Utility package for the University of Udine 
% (e.g. institutional names and translations)
\usepackage{uniud}
% Loads the default configuration for the University of Udine:
%  - affiliation (university and department)
%  - logo
%  - course
% To override the configuration just use '\uniudset' to provide new values.
\uniudsetdefaults

% Document configuration (avoid blank lines)
\uniudset{
	% 'counters by chapter' 
	% - Boolean
	% - Selects the numbering of figures, theorems, equations, etc.
	%   - 'true':  hirearchical numbering by chapter (e.g. Figure 1.2)
	%   - 'false': flat numbering (e.g. Figure 1) 
	% - Defaults to flat numbering
	counters by chapter=true,
	% ---------------------------------------
	% 'content lists style'
	% - Enum: compact, single, double
	% - By default, content lists defined by Memoir's '\newlistof' like the table 
	%   of contents (TOC), the list of figures (LOF), and the list of tables (LOT)
	%   are typeset as chapters. If
	%     content lists style=compact,
	%  then lists do not insert new pages (this is the default behaviour). If
	%     content lists style=single,
	%  then each list starts on a new page. Finally, if
	%     content lists style=double,
	%  then each list behaves like a proper chapter.
	% ---------------------------------------
	% 'affiliation' [mandatory]
	% - List of strings
	% - Describes the candidate affiliation. String with commas must be wrapped
	%   by braces e.g.:
	%     affiliation={
	%       University of Udine, 
	%       {Department of Mathemathics, Computer Science and Phisics},
	%       Laboratory of Models and Applications of Distributed Systems
	%     }
	% - This field is mandatory. Besides setting 'affiliation' directly, the 
	%   class supports the following fields: 
	%     - 'institution'
	%     - 'faculty'
	%     - 'school'
	%     - 'department'
	%     - 'laboratory'
	%   The field 'faculty' prepends 'Faculty of' or 'Facoltà di' (accordingly to 
	%   the main language) to its value. Fields 'school', 'department', and 
	%   'laboratory' work liekwise. Whenever any of these is set, 'affiliation' 
	%   is updated. The order is the same of the list above. The package 'uniud' 
	%   provides localised standard names e.g.: department=\uniudname{dima}.
	% ---------------------------------------
	% 'course' [mandatory]
	% - String
	% - The course name. The utility package 'uniud' provides localised standard %   names e.g.: course=\uniudname{dima\informatica}.
	% ---------------------------------------
	% 'degree' [mandatory]
	% - Enum: bachelor, master, phd (triennale, magistrale, dottorato)
	degree=phd,
	% ---------------------------------------
	% 'title' [mandatory]
	% - String
	% - Title of the thesis
	title={Here goes the title},
	% ---------------------------------------
	% 'subtitle'
	% - String
	% - A subtitle is not mandatory, the field can be omitted.
	subtitle={Here goes the subtitle},
	% ---------------------------------------
	% 'candidate name' [mandatory]
	% - String
	candidate name={Name Surname},
	candidate use gender=feminine,
	candidate contacts={
		\email{candidate@mail}\par
		Dipartimento di Matematica, Informatica e Fisica\\
		Università degli Studi di Udine\\
		Via delle Scienze, 206\\
		33100 Udine\\
		Italia
	},
	% ---------------------------------------
	% 'supervisor name' [mandatory]
	% - String
	% - Supervisor/Relatore
	supervisor name={Name Surname},
	supervisor contacts={
		\email{supervisor@mail}\par
		Dipartimento di Matematica, Informatica e Fisica\\
		Università degli Studi di Udine\\
		Via delle Scienze, 206\\
		33100 Udine\\
		Italia
	},
	% ---------------------------------------
	% 'cosupervisors' 
	% - List of strings
	% - Lists the co-supervisors names, e.g.:
  cosupervisors={
    Name Surname,
    Name Surname
  },
	% ---------------------------------------
	% 'cosupervisor' 
	% - List of strings
	% - This field is a shorthand for 'cosupervisors' where the list is a 
	%   singleton. For instance
	%     cosupervisor={Name Surname},
	%   is equivalent to
	%     cosupervisors={Name Surname},
	% ---------------------------------------
	% 'academic year' [conditional]
	% - String
	% - The field is mandatory whenever 'degree' is either 'bachelor' or 'master'.
	academic year={YYYY/YY},
	% ---------------------------------------
	% 'phd cycle' [conditional]
	%	- String
	% - The field is mandatory whenever 'degree' is set to 'phd' and ignored 
	%   otherwise.
	phd cycle={xxviii},
	% ---------------------------------------
	% 'date'
	% - String
	date=\today,
	% ---------------------------------------
	% 'revision'
	% - String
	revision={Draft},
	% ---------------------------------------
	% 'inside cover note'
	% - String
	% - Anything to be typeset in the inside cover (right after the title page)
	%	inside cover note={}
}

%	BABEL ----------------------------------------------------------------------
\usepackage{babel}
\usepackage[all]{foreign}
\defasforeign[criteria]{criteria}
\defasforeign[criterium]{criterium}

%\hyphenation{}

%	DOCUMENT ORGANIZATION --------------------------------------------------------
\usepackage{subfiles}

\usepackage{currfile}

\ifdefined\docroot\else
	\let\docroot\currfilepath
\fi

\newcommand\IfDocRootTF[2]{%
	\ifcurrfilepath{\docroot}{#1}{#2}%
}
\newcommand\IfDocRootT[1]{\IfDocRootTF{#1}{}}
\newcommand\IfDocRootF[1]{\IfDocRootTF{}{#1}}

%	BIBLATEX ---------------------------------------------------------------------
\usepackage[
	backend=bibtex, %backend=biber
	% numeric citations with compression of consecutive numers
	% e.g. [1-4] instead of [1,2,3,4]
	style=numeric-comp,
	% use given name initials for authors
	giveninits=true,
	% sort references by author names, year and then title
	sorting=nyt,
	% limits the number of displayed authors in the bibliography (default is 3)
	maxbibnames=99,
	% do not print urls or doi and isbn identifiers
	doi=false, 
	isbn=false,
	url=false
	]{biblatex}
\addbibresource{dummy.bib} %biblio sources

%	CROSS REFS -------------------------------------------------------------------
\usepackage{nameref}
\usepackage{hyperref}

\hypersetup{
	% do not highlight hyperlinks such as citations
	hidelinks,
	% link on both page and title in TOC
	linktoc=all
}

% Clever cross references with \cref
\usepackage[
	% capitalise all reference names e.g. Theorem 3
	capitalise,
	% do not use abbreviations such as Fig. for Figure
	noabbrev,
	% include in the hyperlink also the reference name
	nameinlink
]{cleveref}

% THEOREMS ---------------------------------------------------------------------

\DeclareTranslationFallback {fact}{Fact}
\DeclareTranslation{Italian}{fact}{Fatto}
\DeclareTranslationFallback {facts}{Facts}
\DeclareTranslation{Italian}{facts}{Fatti}
\DeclareTheoremName{fact}

\declaretheorem[
  name=\UseTheoremName{fact},
  sibling=theorem
]{fact}

%	TIKZ -------------------------------------------------------------------------
\usepackage{tikz}


% DEMO -------------------------------------------------------------------------
\usepackage{lipsum,blindtext} % generates demo text

%	COMMANDS ---------------------------------------------------------------------


%-------------------------------------------------------------------------------
%	END PREAMBLE
%-------------------------------------------------------------------------------
\begin{document}

%-------------------------------------------------------------------------------
%	FRONT MATTER
%-------------------------------------------------------------------------------
\frontmatter

% Makes the title page
\maketitle

% Includes a dedication page
\begin{dedication}
	Dedication
\end{dedication}

% The environment 'acknowledgements' creates pages for acknowledgements. By 
% default these are not listed in the table of contents (hereafter TOC). 
\begin{acknowledgements}
	\blindtext
\end{acknowledgements}

\begin{otherlanguage}{italian}
\begin{acknowledgements*}
	\blindtext
\end{acknowledgements*}
\end{otherlanguage}

% The environment 'abstract' creates pages for the thesis abstract. By default
% these are not listed in the TOC. 
\begin{abstract}
	\blindtext
\end{abstract}

% Sometimes, a translation of the thesis abstract may be required. The main language can be changed locally by means of babel's 'otherlanguage' environment. (The same applies to acknowledgements.)
\begin{otherlanguage}{italian}
	\begin{abstract*}
		\blindtext
	\end{abstract*}
\end{otherlanguage}

\cleardoublepage

% The commands '\tableofcontents', '\listoffigures', and '\listoftables' typeset, respectively, the table of contents (TOC), the list of figures (LOF), and the list of tables (LOT). All these commands have a starred version (e.g. '\tableofcontents*') which does not add the respective title in the TOC. By default listings like TOC, LOF, and LOT do not always start on odd (or new) pages like ordinary chapters. This behaviour can be changed locally via '\clearpage' and '\cleardoublepage' and globally via the class option 'content lists style'.

\tableofcontents*

\clearpage

\listoffigures*

\listoftables*

%-------------------------------------------------------------------------------
%	MAIN MATTER
%-------------------------------------------------------------------------------

\mainmatter

\book{Book title}

\part{Part title}

\chapter{Chapter title}
\blindmathtrue

\blindtext

\section{Section title}

\blindtext

\subsection{Subsection title}

\blindtext

\begin{equation}
	\label{eq:a}
	a = a
\end{equation}

\begin{equation}
	\label{eq:b}
	\tag{$\dagger$}
	b = b
\end{equation}

\begin{equation}
	\label{eq:c}
	c = c
\end{equation}

\eqref{eq:a}
\eqref{eq:b}


\subsubsection{Subsubsection title}

\blindtext

\paragraph{Paragraph title}

\blindtext

\subparagraph{Subparagraph title}

\blindtext

\section{Lists}


Itemize
\begin{itemize}
\item First item text
\item Second item text
\item Third item text
\end{itemize}
Enumerate
\begin{enumerate}
\item First item text
\item Second item text
\item Third item text
\end{enumerate}
Description
\begin{description}
\item[First] 
	Lorem ipsum dolor sit amet, consectetuer adipiscing elit. Etiam lobortis facilisis sem. Nullam nec mi et neque pharetra sollicitudin. Praesent imperdiet mi nec ante.
\item[Second] 
	 Donec ullamcorper, felis non sodales commodo, lectus velit ultrices augue, a dignissim nibh lectus placerat pede.
\item[Third] 
	Vivamus nunc nunc, molestie ut, ultricies vel, semper in, velit. Ut porttitor. Praesent in sapien.
\end{description}


\section{Math}

\begin{theorem}
	\label{a-theorem}
	A theorem
\end{theorem}

\begin{proof}
	A proof
\end{proof}

\begin{corollary}
	\label{a-corollary}
	A corollary
\end{corollary}

\begin{proposition}
	\label{a-proposition}
	A proposition
\end{proposition}

\begin{lemma}
	\label{a-lemma}
	A lemma
\end{lemma}

\begin{fact}
	\label{a-fact}
	A fact
\end{fact}

\begin{fact}
	\label{another-fact}
	Another fact
\end{fact}

\begin{definition}
	\label{a-definition}
	A definition
\end{definition}

\begin{remark}[A long remark]
	\label{a-remark}
	\blindtext[1]
\end{remark}

\begin{figure}[t]
	\begin{center}
	\begin{tikzpicture}
		\draw (0,0) circle (4cm);
	\end{tikzpicture}
	\end{center}
	\caption{This figure has a very long caption\dots let's see if it comes out nice\dots}
	\label{a-figure}
\end{figure}

\section{Cross referencing and citations}

Here are some references and citations: 
\begin{itemize}
	\item \verb|\cref{a-theorem}| \cref{a-theorem};
	\item \verb|\Cref{a-fact}| \Cref{a-fact};
	\item \verb|\cref{a-fact,another-fact}| \cref{a-fact,another-fact};
	\item \verb|\Cref{a-theorem,a-definition}| \Cref{a-theorem,a-definition};
	\item \verb|\cite{...}| 
	 \cite{article,book,booklet,conference,inbook,incollection,manual,mastersthesis,misc,phdthesis,proceedings,techreport,unpublished}.
\end{itemize}


\IfDocRootTF{A}{B}

\subfile{subfile-example}

\IfDocRootTF{A}{B}

%-------------------------------------------------------------------------------
%	BACK MATTER
%-------------------------------------------------------------------------------

\backmatter

\printbibliography

\appendix

\chapter{Appendix}
\lipsum[1-7]

\end{document}